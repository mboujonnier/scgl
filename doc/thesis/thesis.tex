\documentclass[a4paper,12pt]{thesis}

\autor{Patryk Kwiatkowski}
\tytul{Podstawowa biblioteka grafów~w~C}
\tytulAng{Basic graph library in~C}
\promotor{dr inż. Ireneusz Szcześniak}
\rok{2012}
\kierunek{Informatyka}
\specjalnosc{Sieciowe Technologie Informatyczne}
\numerAlbumu{101510}
\studia{stacjonarne}
\stopien{II}

\newcommand{\eng}[1]{(\emph{#1})}

\begin{document}

\stronaTytulowa

\tableofcontents

\chapter*{Cel pracy}
\addcontentsline{toc}{chapter}{Cel pracy}
Cel i \index{lol}zakres pracy\cite{bib:test}
\chapter*{Wstęp}
\addcontentsline{toc}{chapter}{Wstęp}
\chapter{Wybrane zagadnienia teorii grafów}
\section{Podstawowe pojęcia}
graf, krawędź, węzeł, rodzaje grafów (nieskierwany skierowany), zastosowania grafów, modelowanie sieci 
\section{Algorytm Dijkstry}
pseudkod, po co, złożoność, porównanie do innych (złozoność)
\chapter{Biblioteki w systemach Unix / Linux}
co to, po co 
\section{Statyczne}
zalety / wady jak tworzyc
\section{Współdzielone}
zalety /wady umiejscowienie
\chapter{Realizacja biblioteki}
Do zrealizowania postawionych w~pracy celów zaprojektowano oraz zaimplementowano bibliotekę nazwaną Simple~C Graph Library, dalej określaną skrótem SCGL. 
Projekt ten stworzony został w~oparciu o~język~C~oraz jego bibliotekę standardową (w systemach Unix/Linux: \emph{GNU libc - glibc}).

Wyboru tego dokonano przede wszystkim ze względu na możliwość redukcji wszelkich narzutów wynikających~z cech charakterystycznych dla języków obiektowych (dziedziczenie, polimorfizm, szablony). 
Dodatkowym atutem było bardzo dobre wsparcie kompilatorów oraz szeroki wybór dostępnych narzędzi dla języka C.

Projekty takie jak ten przedstawiony w niniejszej pracy często charakteryzują siędynamicznym rozwojem, zwłaszcza w początkowych fazach tworzenia. 
W~celu zapewnienia poprawności zaimplementowanych już funkcjonalności, zdecydowano się skorzystać~z mechanizmu testów jednostkowych oraz platformy \emph{DejaGNU}.

Dodatkowo biblioteka wykorzystuje program \emph{make} oraz pliki reguł \emph{Makefile} do automatyzacji procesu kompilacji.

Podczas projektowania każdego~z modułów biblioteki wykorzystano wiedzę zawartą w publikacjach TODO, TODO oraz stosowano się do reguły KISS (ang. \emph{Keep It Simple, Stupid}), która traktuje o tym, że im coś jest prostsze (jako koncept, oraz jako wykonanie) tym lepiej TODO (kiss).
\section{Budowa projektu}
\subsection{Diagram klas}
powiązanie klas i opisać każdą z osobna
\subsection{Struktura plików}
/src /include ?
\section{Szczegóły implementacji}
\subsection{Linux Kernel List}
ze kernel, schemat jak działa, ze fajne, ze zmieniłem tak i tak, ze mały narzut, czemu tak - ze dynamiczny rozmiar i ze w C nie ma wektorów i ze trzeba by tablice ciągle realloc co jest wolne
\subsection{Zmienna kosztu? - inaczej nazwać?}
ze w makefile, ze uzytkownik moze dostoswoac, wstawki kodu
\subsection{Algortym Dijkstry}
ze najszybciej na kopcu kolejka prior, ze kolory, jak to u mnie działa i co mozna poprawić ze hash lista
\subsection{Testy jednostkowe - DejaGNU}
co to dejagnu, co to expect i tcl, jak to napisałem czyli wstawka .exp i schemat tests.c (ze switch), co testujemy i po co są testy
\section{Instrukcja użytkownika}
\subsection{Kompilacja}
makefile jak zbudowany, co buduje, co kasuje, ze dokumentacje zbudujemy wchodząc do doc/latex/make 
\subsection{Interfejs programisty - API}
opisać jak są zbudowane funcje ze scgl\_moduł\_funckja, jakie są najważniejsze,ze create/destroy, ze destroy kasuje i ustawia na NULL, jak uzywac atrybutów przykłady foreach 
\chapter{Porównianie z instniejącymi rozwiązaniami}
TODO:opisać co to, i jaką ma filozofie do grafów - wybór padł na boost bo najpopularniejszy i igraph bo równiez jedna z większych bibliotek
\section{Testy porównawcze}
jak testowałem, na jakich kodach wstawki, czasy wyniki pamieciowe wnioski ze da sie szybciej albo czemu cos działa szybciej, szybkosc kompilacji
\chapter*{Podsumowanie}
ze mozna zmniejszyc narzut, ze mozna usprawnić kopiec, ze mozna dodać haslistę  ze mozna dodac algorytmy, 
   ze działą ze dynamiczne ze atrubyty (w igraph to dopiero experymentalna opcja), ze szybkie i w miare małe, 
\addcontentsline{toc}{chapter}{Podsumowanie}
\chapter*{Summary}
\addcontentsline{toc}{chapter}{Summary}
\bibliography{thesis}
\chapter*{Dodatek A. Dokumentacja}
\addcontentsline{toc}{chapter}{Dodatek A. Dokumentacja}
\chapter*{Dodatek B. Oświadczenie}
\addcontentsline{toc}{chapter}{Dodatek B. Oświadczenie}
\chapter*{Dodatek C. Opis zawartości płyty CD}
\addcontentsline{toc}{chapter}{Dodatek C. Opis zawartości płyty CD}

\listoffigures
\listoftables
%\printindex

\end{document}
